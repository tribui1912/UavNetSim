\documentclass[10pt, a4paper]{article}
\usepackage[utf8]{inputenc}
\usepackage{geometry}
\usepackage{graphicx}
\usepackage{hyperref}
\usepackage{listings}
\usepackage{xcolor}
\usepackage{float}
\usepackage{titlesec}
\usepackage{booktabs}
\usepackage{array}
\usepackage{enumitem}

% Page Geometry
\geometry{margin=1in}

% Colors for code listings
\definecolor{codegreen}{rgb}{0,0.6,0}
\definecolor{codegray}{rgb}{0.5,0.5,0.5}
\definecolor{codepurple}{rgb}{0.58,0,0.82}
\definecolor{backcolour}{rgb}{0.95,0.95,0.92}

% Code listing style
\lstdefinestyle{mystyle}{
    backgroundcolor=\color{backcolour},   
    commentstyle=\color{codegreen},
    keywordstyle=\color{magenta},
    numberstyle=\tiny\color{codegray},
    stringstyle=\color{codepurple},
    basicstyle=\ttfamily\footnotesize,
    breakatwhitespace=false,         
    breaklines=true,                 
    captionpos=b,                    
    keepspaces=true,                 
    numbers=left,                    
    numbersep=5pt,                  
    showspaces=false,                
    showstringspaces=false,
    showtabs=false,                  
    tabsize=2
}
\lstset{style=mystyle}

% Hyperlink setup
\hypersetup{
    colorlinks=true,
    linkcolor=blue,
    filecolor=magenta,      
    urlcolor=cyan,
    pdftitle={UavNetSim User Manual},
    pdfpagemode=FullScreen,
}

% Heading formatting
\titleformat{\section}
  {\normalfont\Large\bfseries}{\thesection}{1em}{}
\titleformat{\subsection}
  {\normalfont\large\bfseries}{\thesubsection}{1em}{}
\titleformat{\subsubsection}
  {\normalfont\normalsize\bfseries}{\thesubsubsection}{1em}{}

% Title Page Information
\title{COMPE 560 - Machine Learning\\
Fall 2025\\
User Manual\\
Drone Network Simulator}
\author{Tri Bui\\
RedID: 828135536}
\date{\today}

\begin{document}

% First page (Cover)
\begin{center}
    \vspace*{\fill}
    {\includegraphics[width=5cm]{SDSU_logo.png}}
    \vspace{1cm}

    \Large\textbf{College of Engineering}\\
    \Large\textbf{Department of Electrical and Computer Engineering}\\
    \vspace{1.5cm}

    \LARGE\textbf{COMPE 560}\\
    \Large\textbf{Fall 2025}\\
    \vspace{1.5cm}
    
    \Huge\textbf{Drone Network Simulator}\\
    \vspace{0.5cm}
    \LARGE\textbf{User Manual}\\
    \vspace{2cm}

    \Large\textbf{Group 1}\\
    \vspace{0.5cm}
    \large\textbf{Instructor:} Dr. Umut Can Çabuk
    \vspace*{\fill}
\end{center}
\thispagestyle{empty}
\newpage

% Table of Contents page
\tableofcontents
\thispagestyle{empty}
\newpage

\setcounter{page}{1}

% Content
\section{Introduction}
\textbf{Drone Network Simulator} is a high-fidelity, discrete-event simulator designed for Unmanned Aerial Vehicle (UAV) networks, also known as Flying Ad-hoc Networks (FANETs). It provides a comprehensive environment to model, simulate, and analyze the complex interactions between physical mobility, wireless communication protocols, and energy consumption in 3D space.

\vspace{1em}
\noindent Key capabilities include:
\begin{itemize}
    \item \textbf{Realistic Mobility:} 3D Random Waypoint and Leader-Follower formation flight.
    \item \textbf{Protocol Stack:} Modular implementation of PHY, CSMA/CA MAC, and AODV routing.
    \item \textbf{Energy Modeling:} Detailed power consumption tracking for flight and communication.
    \item \textbf{Interactive GUI:} Real-time 3D visualization with manual controls and live metrics.
\end{itemize}

\section{Installation \& Setup}

\subsection{Option A: The Easy Way (Recommended)}
The Drone Network Simulator comes with a ``zero-setup'' launcher that handles dependencies and virtual environments automatically.

\vspace{0.5em}
\noindent \textbf{Windows Users:}
\begin{enumerate}
    \item Navigate to the \texttt{launcher} folder.
    \item Double-click \texttt{run\_uavnetsim.bat}.
\end{enumerate}

\noindent \textbf{Mac/Linux Users:}
\begin{enumerate}
    \item Open a terminal.
    \item Run: \texttt{./launcher/run\_uavnetsim.sh}\\
    \textit{(Note: You may need to run \texttt{chmod +x launcher/*.sh} first)}
\end{enumerate}

\noindent The launcher provides an interactive menu to:
\begin{itemize}
    \item [1] Run the Simulation (GUI)
    \item [2] Run Unit Tests
    \item [4] Run All Experiments (E1-E3)
    \item [6] Check System Status
\end{itemize}

\subsection{Option B: Manual Installation}
If you prefer to manage the environment yourself:

\subsubsection{Prerequisites}
\begin{itemize}
    \item Python 3.8 or higher
    \item pip package manager
\end{itemize}

\subsubsection{Installation}
\begin{enumerate}
    \item \textbf{Clone the repository:}
    \begin{lstlisting}[language=bash]
git clone https://github.com/tribui1912/UavNetSim.git
cd UavNetSim
    \end{lstlisting}

    \item \textbf{Install dependencies:}
    \begin{lstlisting}[language=bash]
pip install -r requirements.txt
    \end{lstlisting}
    \textit{Key dependencies: \texttt{simpy}, \texttt{pyqt6}, \texttt{pyqtgraph}, \texttt{pyopengl}, \texttt{numpy}, \texttt{pandas}.}
\end{enumerate}

\subsubsection{Running the Simulator (Manual Mode)}
To launch the main GUI application:
\begin{lstlisting}[language=bash]
python main.py
\end{lstlisting}

\noindent To run the automated experiments (headless mode):
\begin{lstlisting}[language=bash]
python experiment_runner.py
\end{lstlisting}

\section{User Interface Guide}
The Drone Network Simulator GUI provides a powerful interface for visualizing network behavior and controlling simulation parameters in real-time.

\begin{figure}[H]
    \centering
    \includegraphics[width=0.9\textwidth]{img/screenshot.png}
    \caption{The Drone Network Simulator Main Interface}
    \label{fig:gui_screenshot}
\end{figure}

\subsection{Main Visualization Area (3D View)}
The central panel displays the 3D simulation environment.
\begin{itemize}
    \item \textbf{UAV Nodes:} Represented as spheres.
    \begin{itemize}
        \item \textbf{Color Coding:} Indicates energy level.
        \begin{itemize}
            \item \textcolor{green}{\textbf{Green:}} $> 50\%$ Energy
            \item \textcolor{yellow}{\textbf{Yellow:}} $20\% - 50\%$ Energy
            \item \textcolor{red}{\textbf{Red:}} $< 20\%$ Energy
        \end{itemize}
    \end{itemize}
    \item \textbf{Links:} Lines drawn between nodes indicate active communication links (neighbors within range).
    \item \textbf{Navigation:}
    \begin{itemize}
        \item \textbf{Rotate:} Left-click and drag.
        \item \textbf{Pan:} Middle-click (or Shift + Left-click) and drag.
        \item \textbf{Zoom:} Scroll wheel.
    \end{itemize}
\end{itemize}

\subsection{Control Panel (Left/Bottom)}
Controls the execution flow of the simulation.
\begin{itemize}
    \item \textbf{Start:} Begins the simulation.
    \item \textbf{Pause:} Temporarily halts the simulation.
    \item \textbf{Reset:} Stops and resets the environment to initial conditions.
    \item \textbf{Speed Slider:} Adjusts the simulation speed multiplier (1x to 10x).
    \item \textbf{Trigger Formation Change:} Manually forces the UAV swarm to switch from Random Waypoint to Leader-Follower formation.
\end{itemize}

\subsection{Statistics Panel (Right)}
Displays real-time Key Performance Indicators (KPIs).
\begin{itemize}
    \item \textbf{Live Metrics:}
    \begin{itemize}
        \item \textbf{PDR (Packet Delivery Ratio):} Percentage of packets successfully delivered.
        \item \textbf{Latency:} Average end-to-end delay (ms).
        \item \textbf{Throughput:} Current network throughput (Kbps).
    \end{itemize}
    \item \textbf{Charts:}
    \begin{itemize}
        \item \textbf{PDR vs Time:} Tracks reliability over the simulation run.
        \item \textbf{Energy Profile:} Shows average residual energy of the swarm.
        \item \textbf{Queue Size:} Monitors buffer occupancy to detect congestion.
    \end{itemize}
\end{itemize}

\subsection{Exporting Data}
\begin{itemize}
    \item \textbf{Export Screenshot:} Saves the current 3D view as a PNG file.
    \item \textbf{Export CSV:} Saves the collected metrics (PDR, Latency, Energy) to a CSV file for external analysis.
\end{itemize}

\section{Simulation Features}

\subsection{Physical Layer}
The PHY layer models the wireless medium and hardware characteristics.
\begin{itemize}
    \item \textbf{Channel Model:} Implements a Log-Distance Path Loss model with probabilistic shadowing. Data loss is simulated with a configurable probability (default 5\%) to mimic fading and interference.
    \item \textbf{Energy Model:} Tracks power consumption in four states:
    \begin{itemize}
        \item \textbf{TX (Transmission):} 1.5 W (configurable)
        \item \textbf{RX (Reception):} 1.0 W
        \item \textbf{Idle:} 0.1 W
        \item \textbf{Sleep:} 0.001 W
        \item \textbf{Initial Energy:} 200 kJ per drone (enables $\sim$180s flight at 10 m/s)
    \end{itemize}
    \item \textbf{Flight Power:} $\sim$1100 W at 10 m/s, calculated using aerodynamic principles (Zeng et al. model). Flight power dominates total energy consumption.
\end{itemize}

\subsection{MAC Layer}
The Link Layer manages access to the shared medium.
\begin{itemize}
    \item \textbf{Protocol:} CSMA/CA (Carrier Sense Multiple Access with Collision Avoidance).
    \item \textbf{Features:}
    \begin{itemize}
        \item \textbf{Backoff Mechanism:} Exponential backoff with configurable contention window ($CW_{min}=31$).
        \item \textbf{Reliability:} Stop-and-Wait ARQ with ACKs.
        \item \textbf{Retransmissions:} Packets are dropped after \texttt{MAX\_RETRANSMISSION\_ATTEMPT} (5) failures.
        \item \textbf{Queuing:} Per-node FIFO queues with finite capacity.
    \end{itemize}
\end{itemize}

\subsection{Network Layer}
Handles routing and end-to-end packet delivery.
\begin{itemize}
    \item \textbf{Protocol:} AODV (Ad hoc On-Demand Distance Vector).
    \item \textbf{Mechanism:} Reactive routing. Routes are discovered only when needed via RREQ (Route Request) and RREP (Route Reply) cycles.
    \item \textbf{Maintenance:}
    \begin{itemize}
        \item \textbf{Hello Packets:} Periodic beacons (1 Hz) to maintain neighbor tables.
        \item \textbf{Link Breaks:} Detected via MAC layer feedback (ACK timeout), triggering RERR (Route Error) packets and local repair.
        \item \textbf{Buffering:} Packets are buffered at the source while waiting for route discovery.
    \end{itemize}
\end{itemize}

\subsection{Mobility Models}
Defines how UAVs move in the 3D space.
\begin{itemize}
    \item \textbf{3D Random Waypoint:} Nodes select random destinations within the $600 \times 600 \times 100$ m volume, move at constant speed, pause, and repeat.
    \item \textbf{Leader-Follower Formation:} A designated leader follows a path (e.g., RWP), while follower nodes maintain fixed relative offsets (e.g., V-formation).
    \item \textbf{Dynamic Switching:} The simulator supports switching mobility models mid-run (e.g., at $t=300s$) to test protocol adaptability.
\end{itemize}

\section{Running Experiments}
The \texttt{experiment\_runner.py} script automates the execution of predefined scenarios.

\subsection{E1: Mobility Model Comparison}
\begin{itemize}
    \item \textbf{Goal:} Compare Leader-Follower formation vs Random Waypoint at different network scales.
    \item \textbf{Setup:} N = 5, 25, 100 drones with tiered durations (50s for N=5,25; 30s for N=100).
    \item \textbf{Finding:} Formation provides 30\% latency reduction in small networks but both models degrade at scale due to congestion.
\end{itemize}

\subsection{E2: TX Power vs Lifetime/PDR}
\begin{itemize}
    \item \textbf{Goal:} Analyze the impact of transmission power on network performance.
    \item \textbf{Setup:} 25 drones, TX power varying from 0.5-2.5 W, 600s max duration.
    \item \textbf{Finding:} Lifetime is flight-dominated ($\sim$181s regardless of TX power). Higher TX power improves PDR (11.5\% at 0.5W $\to$ 25.3\% at 2.5W) by extending range.
\end{itemize}

\subsection{E3: Formation Transition}
\begin{itemize}
    \item \textbf{Goal:} Assess network stability and recovery during topology changes.
    \item \textbf{Setup:} 25 drones, 200s simulation. Switch from Random Waypoint to Formation flight at $t=100s$.
    \item \textbf{Finding:} Network recovered in 1 second with improved stability (PDR increased from 20\% to 24\% post-transition).
\end{itemize}

\section{Statement of Work}
This project was collaboratively developed by a team of 10 members. The responsibilities were divided to ensure coverage of all system components, from low-level physical modeling to high-level application GUI and analysis.

\begin{table}[H]
    \centering
    \renewcommand{\arraystretch}{1.5}
    \begin{tabular}{|p{0.25\linewidth}|p{0.25\linewidth}|p{0.4\linewidth}|}
        \hline
        \textbf{Member Name} & \textbf{Role} & \textbf{Primary Contributions} \\
        \hline
        [Name 1] & Project Lead \& Architect & System architecture design, core event engine (SimPy) integration, overall project management. \\
        \hline
        [Name 2] & PHY Layer Lead & Implementation of Channel models (Path Loss, Fading) and SINR calculations. \\
        \hline
        [Name 3] & Energy Model Specialist & Development of the Energy Consumption Model (Flight + Comm power) and battery state logic. \\
        \hline
        [Name 4] & MAC Layer Developer & Implementation of CSMA/CA protocol, Backoff logic, and ACK/Retransmission mechanisms. \\
        \hline
        [Name 5] & Network Layer Lead & Implementation of AODV Routing Protocol (RREQ, RREP, RERR handling) and routing tables. \\
        \hline
        [Name 6] & Mobility Model Developer & Implementation of 3D Random Waypoint and Leader-Follower formation logic. \\
        \hline
        [Name 7] & GUI \& Visualization & Development of the PyQt6/OpenGL 3D visualization, rendering pipeline, and interactive controls. \\
        \hline
        [Name 8] & Data Analysis \& Metrics & Implementation of the statistics collection system, real-time plotting, and CSV/PNG export features. \\
        \hline
        [Name 9] & Experiment Runner & Creation of \texttt{experiment\_runner.py}, automation of E1/E2/E3 scenarios, and result aggregation. \\
        \hline
        [Name 10] & QA \& Documentation & Comprehensive testing, bug tracking, and compilation of the User Manual and Design Document. \\
        \hline
    \end{tabular}
    \caption{Statement of Work and Contribution Breakdown}
    \label{tab:sow}
\end{table}

\section{References}
\begin{enumerate}[label={[\arabic*]}]
    \item Z. Zhou et al., ``UavNetSim-v1: A Python-based Simulation Platform for UAV Communication Networks,'' \textit{arXiv preprint arXiv:2507.09852}, 2025.
    \item C. Perkins, E. Belding-Royer, and S. Das, ``Ad hoc On-Demand Distance Vector (AODV) Routing,'' RFC 3561, 2003.
    \item Y. Zeng, J. Xu, and R. Zhang, ``Energy Minimization for Wireless Communication with Rotary-Wing UAV,'' \textit{IEEE Transactions on Wireless Communications}, vol. 18, no. 4, pp. 2329-2345, 2019.
    \item IEEE Standard for Information technology—Telecommunications and information exchange between systems Local and metropolitan area networks—Specific requirements - Part 11: Wireless LAN Medium Access Control (MAC) and Physical Layer (PHY) Specifications, \textit{IEEE Std 802.11-2016}.
\end{enumerate}

\end{document}
